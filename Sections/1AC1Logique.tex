%%%                         Louis-Thibault GAUTHIER                       %%%


\section{Logique}

\subsection{Connecteurs logiques}



Revenez 1 min à la définition \ref{Defassertion} page \pageref{Defassertion}.


\Def[(Négation)]
{
	Soit $P$ et $Q$ deux propositions. \\
	La \textbf{\textcolor{gold2}{négation}} \index{Négation} de $P$ est la proposition, notée \textbf{non $P$}, ou encore $\overline{P}$,  définie par :
	\begin{itemize}
		\item non $P$ est vraie lorsque $P$ est fausse.
		\item non $P$ est fausse lorsque $P$ est vraie.
	\end{itemize}
}
Donnons sa table de vérité :
\begin{center}
	\begin{tabular}{|c||c|}
		\hline
		\textbf{$P$}  & \textbf{$\overline{P}$}  \\
		\hline
		V & F \\
		\hline
		F & V \\
		\hline
	\end{tabular}
\end{center}

\Def[(Et \& Ou)]
{
	La \textbf{\textcolor{gold2}{conjonction}} de $P$ et de $Q$ est la proposition, notée \textbf{$P$ et $Q$}, ou encore $P \land Q$, définie de la manière suivante :
	\begin{itemize}
		\item $P$ et $Q$ est vraie lorsque $P$ et $Q$ sont vraies.
		\item $P$ et $Q$ est fausse lorsqu'au moins une des deux propositions est fausse.
	\end{itemize}
	La \textbf{\textcolor{gold2}{disjonction}}  \index{Disjonction} de $P$ et $Q$ est la proposition, notée \textbf{$P$ ou $Q$},  ou encore $P \lor Q$, définie de la manière suivante :
	\begin{itemize}
		\item $P$ ou $Q$ est vraie lorsque l'une au moins des deux propositions est vraie.
		\item $P$ ou $Q$ est fausse lorsque $P$ et $Q$ sont fausses.
	\end{itemize}
}

\newpage

Leurs tables de vérité sont :

\begin{center}
	\begin{multicols}{2}
			\begin{tabular}{|c|c||c|}
				\hline
				\textbf{$P$}  & \textbf{$Q$} & \textbf{$P \land Q$} \\
				\hline
				V & V & \\
				\hline
				V& F &  \\
				\hline
				F& V &  \\
				\hline							
				F& F &  \\
				\hline
			\end{tabular}
			\begin{tabular}{|c|c||c|}
				\hline
				\textbf{$P$}  & \textbf{$Q$} & \textbf{$P \lor Q$} \\
				\hline
				V & V & \\
				\hline
				V& F &  \\
				\hline
				F& V &  \\
				\hline							
				F& F &  \\
				\hline
			\end{tabular}
	\end{multicols}	 	
\end{center}


\subsection{Implication}




\Def{
	Soit $P$ et $Q$ deux propositions. \\
	\textbf{\textcolor{gold2}{L'implication}} $P$ entraîne $Q$ est la proposition, notée \textbf{$P \Rightarrow Q$}  est définie par \textcolor{gold2}{(non $P$ ou $Q$)}. \\
	\textbf{\textcolor{gold2}{L'équivalence}} de $P$ et de $Q$ est la proposition, notée \textbf{$P \Leftrightarrow Q$}, définie par la conjonction de $P \Rightarrow Q$ et $Q \Rightarrow P$.
}


\Exo{
Donner leurs tables de vérité.
}

\begin{proof}
	Remplissons les tableaux suivants : 
	\begin{center}
		\begin{tabular}{|c|c||c||c|}
			\hline
			\textbf{$P$}  & \textbf{$Q$} &   $\overline{P}$ &$\overline{P} \lor Q  :P \Rightarrow Q$ \\
			\hline
			V & V &  &\\
			\hline
			V& F &  & \\
			\hline
			F& V &  &  \\
			\hline							
			F& F &  & \\
			\hline
		\end{tabular}
	\end{center}
	\begin{center}
		\begin{tabular}{|c|c||c|c|c|c||c| }
				\hline
				\textbf{$P$}  & \textbf{$Q$} & $\overline{P}$ &  $\overline{Q}$  & $ P \Rightarrow Q$ &$ Q \Rightarrow P$ & $P \Leftrightarrow Q$ \\
				\hline
				V & V & & & & & \\
				\hline
				V& F & & & & &   \\
				\hline
				F& V   & & & &  & \\
				\hline							
				F& F & & & & &   \\
				\hline
			\end{tabular}
	\end{center}
\end{proof}

\Def{
	Soit $P$ et $Q$ deux propositions. 
	\begin{itemize}
		\item Lorsque $P \Rightarrow Q$ est vraie, on dit que $P$ est une \textcolor{gold2}{\textbf{condition suffisante}} pour avoir $Q$, et que $Q$ est une  \textcolor{gold2}{\textbf{condition nécessaire}} pour avoir $P$.
		\item Lorsque $P \Leftrightarrow Q$ est vraie, $P$ est une  \textbf{\textcolor{gold2}{condition nécessaire et suffisante}} pour avoir $Q$.
	\end{itemize}
}


\begin{multicols}{2}
	\Def[(Autres formulations)]
	{
		Soit $P$ et $Q$ deux assertions. \\
		Au lieu de dire on a  \og  $P \Rightarrow Q$ \fg, on peut dire :
		\begin{itemize}
			\item Pour $Q$ soit vraie, il suffit que $P$ le soit.
			\item Pour que $P$ soit vraie, il faut que $Q$ le soit.
			\item $P$ est une condition suffisante pour que $Q$ soit vraie.
			\item $Q$ est une condition nécessaire pour que $P$ soit vraie.
		\end{itemize}
		Au lieu de dire on a  \og  $P \Leftrightarrow Q$ \fg, on peut dire :
		\begin{itemize}
			\item $P$ est vraie si et seulement si $Q$ l'est.
			\item Pour que $Q$ soit vraie, il faut et il suffit que $P$ le soit.
			\item $P$ est une condition nécessaire et suffisante pour que $Q$ soit vraie.
		\end{itemize}
	}
	\Rem
	{
		Considérons l'implication : 
		\begin{center}
			Je suis à Lyon $\Longrightarrow$ Je suis en France.
		\end{center}
		Il est clair qu'elle est vraie, et on a :
		\begin{itemize}
			\item Pour être en France, il suffit d'être à Lyon.
			\item Pour être à Lyon, il est nécessaire d'être en France.
			\item Il est suffisant d'être à Lyon pour être en France.
			\item Il est nécessaire d'être en France pour être à Lyon.
		\end{itemize}
	}
\end{multicols}


\subsection{Négation}


\Prop[(Négation des propositions avec quantificateurs)]  
{
	Soit $P(x)$ une propriété dépendant d'un paramètre $x$, où $x$ est un élément d'un ensemble $E$. 
	\begin{enumerate}
		\item La négation de la proposition $\forall x \in E, P(x)$ est : $\exists x \in E, \mbox{non } P(x)$.
		\item La négation de la proposition $\exists x \in E, P(x)$ est : $\forall x \in E, \mbox{non } P(x)$.
	\end{enumerate}
}

\example
	Soit $(u_n)_n$ une suite réelle. Donner la négation de la proposition suivante :
	\[	
		\exists M \in \RR, \forall n \in \NN, u_n \leq M
	\]
Cette dernière signifie que la suite  $(u_n)_n$ est bornée.

\Rap
{
	Une proposition est fausse si et seulement si sa négation est vraie.
}

\example 
	\begin{enumerate}
		\item Soit $f\in \RR^{\RR}$. Que penser de l'assertion suivante (vraie, fausse ?)
		\[	
			\forall x \in \RR \quad \left( f(x) \geq 0\lor f(x) \leq 0 \right) \quad ?
		\]
		
		\item De même, que penser de l'assertion 
		\[	
			\forall f\in \RR^{\RR} \left( \forall x \in \RR ;  f(x) \geq 0  \right) \lor \left( \forall x' \in \RR ; f(x') \leq 0  \right) \quad  ?
		\]
	\end{enumerate}




%%%                         Louis-Thibault GAUTHIER                       %%%

