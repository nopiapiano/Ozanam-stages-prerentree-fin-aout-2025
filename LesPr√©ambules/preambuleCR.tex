\documentclass[a4paper,10pt,twoside,landscape,twocolumn]{article}


\usepackage{amssymb}
\setlength{\columnsep}{1cm}
\usepackage{multicol}
\usepackage{amsthm}
\usepackage{eurosym}
\usepackage{graphics}
\usepackage{amsmath}
% For Landau's notation
\usepackage{mismath}

\usepackage{stmaryrd} % Where \llbracket and \rrbracket are defined

\usepackage{parallel}

\usepackage{hyperref}

\let\oldfootnotemark\footnotemark
\renewcommand{\footnotemark}{%
	\begin{NoHyper}\oldfootnotemark\end{NoHyper}%
}
\usepackage[french]{babel}
\usepackage[utf8]{inputenc}  
% \usepackage[latin1]{inputenc}
\usepackage[T1]{fontenc}  
\usepackage{pstricks-add}
\usepackage{graphicx}
\usepackage[cyr]{aeguill} % Guillemets en francais et autres
\usepackage{babel,varioref,amssymb,float} % polices
\usepackage{pgf,tikz} % permet de créer comme pstricks des figures en code LateX
\let\clipbox\relax

\usetikzlibrary{calc,decorations.pathmorphing,arrows}

\usepackage{tikz-3dplot}

\usepackage{vwcol}  % Pour la barre verticale


\usepackage{pgflibraryarrows} % librairie liée à tikz
\usepackage{pgflibrarysnakes}
\usepackage{xcolor} % module de couleur pour tikz
\usepackage{pst-eucl}
\usepackage{titlesec} % Personnaliser les titres
\usepackage{graphicx}
\usepackage[export]{adjustbox}

\usepackage{amsmath,amsfonts,amssymb,theorem,graphicx,geometry}
\geometry{top=1.8cm,left=1cm,right=1cm,bottom=1.5cm}%,marginparwidth=2cm,marginparsep=0.5cm}

\usepackage{multicol}

\usepackage{enumitem}

\usepackage{calc,color}
\usepackage{fancybox}
\usepackage{lastpage}
\usepackage{fancyhdr}
\usepackage{multicol}
\usepackage{marginnote}
\usepackage{fontawesome}

\usepackage[weather]{ifsym}
\usepackage{pifont}

\usepackage{tcolorbox}

\usepackage{thmtools}
 \usepackage[tikz]{bclogo} 
 
 \usepackage{pgfplots} %Diagrammes
 
 
 
% \usepackage{hyperref}


\setlist[itemize]{label=\textbullet}
 
 \usepackage[mmddyyyy]{datetime}% http://ctan.org/pkg/datetime
 \usepackage{advdate}% http://ctan.org/pkg/advdate
 \newdateformat{syldate}{\twodigit{\THEDAY}/\twodigit{\THEMONTH}/\THEYEAR}
 \newsavebox{\MONDAY}\savebox{\MONDAY}{Mon}% Mon
 \newcommand{\week}[1]{%
 %  \cleardate{mydate}% Clear date
 % \newdate{mydate}{\the\day}{\the\month}{\the\year}% Store date
   \paragraph*{\kern-2ex\quad #1, \syldate{\today} - \AdvanceDate[4]\syldate{\today}:}% Set heading  \quad #1
 %  \setbox1=\hbox{\shortdayofweekname{\getdateday{mydate}}{\getdatemonth{mydate}}{\getdateyear{mydate}}}%
   \ifdim\wd1=\wd\MONDAY
     \AdvanceDate[7]
   \else
     \AdvanceDate[7]
   \fi%
 }
 

\def\marginsymbol{\protect\marginsymbolhelper}
\def\marginsymbolhelper{\tabto*{-1cm}\makebox[0cm]{$\bullet$}\tabto*{\TabPrevPos}}

\renewcommand\thesubsection{\Alph{subsection}}
\renewcommand{\thesection}{\Roman{section}}
%\newcommand{\NN}{\mathbb{N}}
%\newcommand{\RR}{\mathbb{R}}

% Small bullet points
\newcommand{\smtextbullet}{\,\begin{picture}(-1,1)(-1,-3)\circle*{2}\end{picture}\ }


%Faire des flèches avec Tikz
\newcommand{\tikzmark}[1]{\tikz[overlay,remember picture] \node (#1) {};}


% Albert School color theme
\definecolor{albertBlue1}{RGB}{34, 20, 81}
\definecolor{albertBlue2}{RGB}{54, 37, 157}
\definecolor{albertBlue3}{RGB}{173, 218, 247}

\definecolor{inter}{RGB}{156, 66, 167}

%%% Les couleurs %%%
\definecolor{fondtitre}{rgb}{0.20,0.43,0.09}  % vert fonce
\definecolor{gold}{RGB}{223,179,85}  % Gold
%\definecolor{gold2}{RGB}{223,179,85} % Gold
\definecolor{gold2}{RGB}{53,93,173}   % Gold2 %bleu
\definecolor{gold3}{RGB}{255,214,100} 
\definecolor{goldinv}{RGB}{32,76,170}
\definecolor{bleu}{RGB}{53,93,173}  
\definecolor{vert}{RGB}{155,205,50}  
\definecolor{vertfoncé}{RGB}{0,128,32}  
\definecolor{bleufoncé}{RGB}{0,32,128}  
\definecolor{grenat}{RGB}{110,11,20}  
\definecolor{amande}{RGB}{130,196,108} 
\definecolor{magentapink}{RGB}{204,51,139}   
\definecolor{glauque}{RGB}{100,155,136}   
\definecolor{oeuf}{RGB}{239,213,178}   
\definecolor{bisque}{RGB}{255,228,196}   
\definecolor{inter}{RGB}{156, 66, 167}
%\definecolor{backg}{RGB}{185,220,200}   
%\definecolor{backg}{RGB}{215,224,240}
\definecolor{crème}{RGB}{253,247,196}  
\definecolor{noirleger}{RGB}{79,77,81}  
\definecolor{marron}{RGB}{88,41,0} 
\definecolor{gris}{RGB}{150,150,150} 
\definecolor{rougem}{RGB}{200,0,0} 
\definecolor{vertm}{RGB}{0,200,0} 
\definecolor{bleum}{RGB}{0,0,200} 


\newcommand{\isEquivTo}[1]{\underset{#1}{\sim}}

\newcommand{\poubelle}[1]{} % Commande Poubelle : "\poubelle{ blabla}" et le blabla sera ignoré, même si il tient en 200 lignes

% Compteur commun
\newcounter{envcounter}

% Properties
\definecolor{propColor}{RGB}{0,128,32}
\newcommand{\Props}[2][]{
	\refstepcounter{envcounter}
	\begin{tcolorbox}[colback=propColor!5!white,colframe=propColor!75!black,title=Propositions \theenvcounter\ #1]
		#2
\end{tcolorbox}}

\definecolor{Cor}{RGB}{0,96,24}
\newcommand{\Cor}[2][]{
	\refstepcounter{envcounter}
	\begin{tcolorbox}[colback=Cor!5!white,colframe=Cor!75!black,title=Corollary \theenvcounter\ #1]
		#2
\end{tcolorbox}}

\newcommand{\Prop}[2][]{
	\refstepcounter{envcounter}
	\begin{tcolorbox}[colback=propColor!5!white,colframe=propColor!75!black,title=Proposition \theenvcounter\ #1]
		#2
\end{tcolorbox}}

% Definitions
\definecolor{defColor}{RGB}{54, 37, 157}
\newcommand{\Defs}[2][]{
	\refstepcounter{envcounter}
	\begin{tcolorbox}[colback=defColor!5!white,colframe=defColor!75!black,title=Definitions \theenvcounter\ #1]
		#2
\end{tcolorbox}}
\newcommand{\Def}[2][]{
	\refstepcounter{envcounter}
	\begin{tcolorbox}[colback=defColor!5!white,colframe=defColor!75!black,title=Définition \theenvcounter\ #1]
		#2
\end{tcolorbox}}


\definecolor{defpropColor}{RGB}{155,205,50}  
\newcommand{\Defprop}[2][]{
\refstepcounter{envcounter}
\begin{tcolorbox}[colback=defpropColor!5!white,colframe=defpropColor!75!black,title=Définition-Proposition \theenvcounter\ #1]
#2
\end{tcolorbox}}


% Theorems
\definecolor{theoremColor}{RGB}{200,0,0} 
\newcommand{\Thm}[2][]{
	\refstepcounter{envcounter}
	\begin{tcolorbox}[colback=theoremColor!5!white,colframe=theoremColor!75!black,title=Théorème \theenvcounter\ #1]
		#2
\end{tcolorbox}}

\definecolor{defRapColor}{RGB}{200,110,0} 
\newcommand{\Rap}[2][]{
	\refstepcounter{envcounter}
	\begin{tcolorbox}[colback=defRapColor!5!white,colframe=defRapColor!75!black,title=Rappel\theenvcounter\ #1]
		#2
\end{tcolorbox}}

\definecolor{defremColor}{RGB}{88,41,0} 
\newcommand{\Rem}[2][]{
	\refstepcounter{envcounter}
	\begin{tcolorbox}[colback=defremColor!5!white,colframe=defremColor!75!black,title=Remarque \theenvcounter\ #1]
		#2
\end{tcolorbox}}

\definecolor{delemColor}{RGB}{54, 37, 157}
\newcommand{\Lem}[2][]{
	\refstepcounter{envcounter}
	\begin{tcolorbox}[colback=delemColor!5!white,colframe=delemColor!75!black,title=Lemme \theenvcounter\ #1]
		#2
\end{tcolorbox}}


% Exo
\definecolor{exoColor}{RGB}{173, 218, 247}
\newcommand{\Exo}[2][]{
	\refstepcounter{envcounter}
	\begin{tcolorbox}[colback=exoColor!5!white,colframe=exoColor!75!black,title=Exercise \theenvcounter\ #1]
		#2
\end{tcolorbox}}

% Examples
\newcommand{\example}[0]{
	\vspace{1em}
	{\bf Exemple :}
}

% Diverse things to highlight
\newcommand{\highlight}[1]{
	\vspace{1em}
	{\bf #1}
	\vspace{0.25em}
	
}

\definecolor{gold}{RGB}{223,179,85}  % Gold


	\newcommand\Mysection[1]{\color{bleu}\section[#1]{\textsc{\color{bleu}#1}}\color{black}}
	\newcommand\Mysubsection[1]{\color{bleu}\subsection[#1]{\textsc{\color{bleu}#1}}\color{black}}
	%\renewcommand{\thesubsubsection}{\color{gold}\arabic{gold}}
	
	
	%%% Couleurs des théorèmes
	\newtheoremstyle{thmgold}{\topsep}{\topsep}%
	{\itshape}% Body font
	{}% Indent amount (empty = no indent, \parindent = para indent)
	{\scshape}% Thm head font
	{\textcolor{gold}{.}}% Punctuation after thm head
	{ }% Space after thm head (\newline = linebreak)
	{\textcolor{gold}{\thmname{#1}\thmnumber{~#2}\thmnote{ (\normalfont #3)}}}% Thm head spec
	
	\theoremstyle{thmgold}
	
	% compteur d'exercices : à mettre entre \begin{exo} et \end{exo}
	\newcounter{num}
	\newcommand\exo{\stepcounter{num}\vspace{1em}\noindent{\textcolor{albertBlue2}{\underline{\bf\large Exercice~\thenum}}}\vspace{0.2em}}
	
	% compteur d'exercices : à mettre entre \begin{exo} et \end{exo}
	\newcounter{sol}
	\newcommand\sol{\stepcounter{sol}\vspace{1em}\noindent{\underline{\bf\large Solution~\thesol}}\vspace{0.2em}}
	%%%
	
\usetikzlibrary{matrix,positioning}

%titre
\newcommand{\titre}[1]{\begin{tikzpicture}
		\matrix [matrix of nodes]{
			|[draw=albertBlue3,very thick,font=\Large,rounded corners,inner sep=0.2cm, fill=albertBlue3]|{\textsc{\ #1\ }}  \\
		} ;
\end{tikzpicture}}





\newcommand{\bigslant}[2]{{\raisebox{.1em}{$#1$}\left/\raisebox{-.2em}{$#2$}\right.}} % Symbole quotient oblique exemple \bigslant{\Z}{n\Z}
\newcommand{\calG}{\mathcal{G}} % Les symboles de maths 
\newcommand{\calA}{\mathcal{A}}
\newcommand{\calU}{\mathcal{U}}
\newcommand{\calS}{\mathcal{S}}
\newcommand{\calF}{\mathcal{F}}
\newcommand{\calC}{\mathcal{C}}
\newcommand{\calB}{\mathcal{B}}
\newcommand{\calD}{\mathcal{D}}
\newcommand{\calE}{\mathcal{E}}
\newcommand{\calM}{\mathcal{M}}
\newcommand{\calN}{\mathcal{N}}
\newcommand{\calH}{\mathcal{H}}
\newcommand{\calI}{\mathcal{I}}
\newcommand{\calL}{\mathcal{L}}
\newcommand{\calR}{\mathcal{R}}
\newcommand{\calP}{\mathcal{P}}
\newcommand{\calO}{\mathcal{O}}
\newcommand{\calT}{\mathcal{T}}
\newcommand{\calV}{\mathcal{V}}
\newcommand{\EE}{\mathbb{E}}
\newcommand{\RR}{\mathbb{R}}
\newcommand{\PP}{\mathbb{P}}
\newcommand{\jP}{\mathbb{P}}
\newcommand{\KK}{\mathbb{K}}
\newcommand{\CC}{\mathbb{C}}
\newcommand{\ZZ}{\mathbb{Z}}
\newcommand{\NN}{\mathbb{N}}
\newcommand{\UU}{\mathbb{U}}
\newcommand{\QQ}{\mathbb{Q}}
\newcommand{\FF}{\mathbb{F}}
\newcommand{\bbP}{\mathbb{P}}
\newcommand{\OFP}{(\Omega,\calF,\bbP)}
\newcommand{\Pa}{\mathscr{P}}
\newcommand{\Oa}{\mathscr{O}}
\newcommand{\Ta}{\mathscr{T}}
\newcommand{\Sa}{\mathscr{S}}
\newcommand{\car}{\mathds{1}} %Les symboles de maths 
%%% --------------------------------------- %%%






\newcommand{\pifont}[1]{%
	\marginnote{\hspace{-0cm}\scalebox{4}{\ding{#1}}}%
}

\newcommand{\fatcoeur}{%
	\marginnote{%
		\hspace{0cm}\scalebox{1}{\bccoeur}%
	}%
}






\poubelle{
\newcommand{\pifont}[1]{%
	\reversemarginpar
	\marginpar{\hspace{-0.3cm}\scalebox{4}{\ding{#1}}}%
}


\newcommand{\fatcoeur}{%
	\marginnote{%
		\hspace{-0.5cm}\scalebox{1}{\bccoeur}%
	}%
}

\newcommand{\fatcoeur}{%
	\marginpar[\hspace{1cm}\scalebox{2}{\bccoeur}]% left margin
	{\hspace{1cm}\scalebox{2}{\bccoeur}}% right margin
}


}



\usepackage[export]{adjustbox}
\DeclareUnicodeCharacter{20AC}{\ifmmode\text{\euro}\else\euro\fi}